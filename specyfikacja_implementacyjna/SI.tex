\documentclass[10pt,a4paper]{article}
\usepackage[utf8]{inputenc}
\usepackage{amsmath}
\usepackage{amsfonts}
\usepackage{amssymb}
\usepackage{polski}
\usepackage{latexsym}
\usepackage{enumitem}
\usepackage{graphicx}

\usepackage{lastpage}
\usepackage{fancyhdr}
\pagestyle{fancy}



\fancyhf{}
\renewcommand{\headrulewidth}{0pt}
\cfoot{Strona \thepage\ z \pageref{LastPage}}


\title{\huge AiSD - laboratorium \\ \Large Projekt zespołowy - specyfikacja implementacyjna}
\author{Kacper Baczyński, Michał Kiełczykowski, Marek Knosala, \\ Edward Sucharda}

\begin{document}

\maketitle

\section{Wstęp}

\section{Opis struktury projektu}

\subsection{Założenia wstępne}

//Opis kolejnych postawionych problemów w zadaniu (narysowanie mapy, dodawanie pacjentów, stworzenie i przeszukiwanie grafu itp)

\subsection{Wykorzystane technologie}

//Java + JavaFX jakiś opis

\subsection{Diagram klas}

\section{Problemy Algorytmiczne}

\subsection{Znalezienie obrysu państwa}

//Problem Convex hull

\subsection{Wykrycie skrzyżowań}

//Naive or Bentley–Ottmann algorithm

\subsection{Określenie czy pacjent znajduje się w obszarze Państwa}

//Even-odd rule Algorithm

\subsection{Przeszukiwanie grafu w celu znalezienia miejsca w szpitalu}

//Zmodyfikowany DFS/BFS

\section{Testy oprogramowania}


\section{Źródła}

\end{document}{article}


