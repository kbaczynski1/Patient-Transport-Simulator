\documentclass[10pt,a4paper]{article}
\usepackage[utf8]{inputenc}
\usepackage{amsmath}
\usepackage{amsfonts}
\usepackage{amssymb}
\usepackage{polski}
\usepackage{latexsym}
\usepackage{enumitem}
\usepackage{graphicx}

\usepackage{lastpage}
\usepackage{fancyhdr}
\pagestyle{fancy}



\fancyhf{}
\renewcommand{\headrulewidth}{0pt}
\cfoot{Strona \thepage\ z \pageref{LastPage}}


\title{\huge AiSD - laboratorium \\ \Large Projekt zespołowy - specyfikacja implementacyjna}
\author{Kacper Baczyński, Michał Kiełczykowski, Marek Knosala, \\ Edward Sucharda}

\begin{document}

\maketitle

\section{Wstęp}

\section{Opis struktury projektu}

\subsection{Założenia wstępne}

//Opis kolejnych postawionych problemów w zadaniu (narysowanie mapy, dodawanie pacjentów, stworzenie i przeszukiwanie grafu itp)

\subsection{Wykorzystane technologie}

//Java + JavaFX jakiś opis

\subsection{Diagram klas}

\section{Problemy Algorytmiczne}

\subsection{Znalezienie obrysu państwa}

//Problem Convex hull

\subsection{Wykrycie skrzyżowań}

//Naive or Bentley–Ottmann algorithm


W postawionym zadaniu istnieje założenie, że jeżeli drogi przecinają się to w miejscu przecięcia powstaje skrzyżowanie.
Powoduje to, że od pewnego szpitala do innego szpitala można dojechać okrężną, lecz szybszą drogą, mimo że nie istnieje ona w pliku wejściowym.
Biorąc pod uwagę, że wszystkie obiekty mapy posiadają współrzędne kartezjańskie, punkty przecięć można wyznaczyć w sposób czysto matematyczny.
Przedstawiając drogę od szpitala do szpitala jako odcinek, można przedstawić wszystkie drogi jako odcinki i znaleźć ich punkty przecięcia.
Jest to metoda naiwna, ponieważ wymaga ona przeanalizowania każdej drogi z każdą inną drogą.
O ile dla jednej pary złożoność obliczeniowa to O(1), tak dla n dróg jest to już O($n^2$).

Istnieje także bardziej przemyślany algorytm, który poprzez "omiecenie wiązką" przez wszystkie odcinki jest w stanie wykryć ich punkty przecięć.
Implementacja algorytmu Bentley–Ottmann wymaga przedstawienia dróg jako posortowanych punktów względem osi X oraz odcinków. Algorytm można przedstawić następująco:
\begin{itemize}
    \item Pionowa linia "przemiata" wszystkie punkty od lewej do prawej.
    \item Po natknięciu się na lewy(początkowy) punkt odcinka, odcinek oznaczany jest jako aktywny.
    \item Następnie sprawdzane są przecięcia z najbliższymi, aktywnymi odcinkami powyżej i poniżej aktualnego odcinka.
    \item Jeżeli odcinki aktywne się przecinają wtedy jest znajdowany i zapamiętywany punkt przecięcia między tymi odcinkami.
    \item W momencie gdy pionowa linia napotka końcowy punkt odcinka, wtedy dezaktywuje ona odcinek. Sprawia to, że odcinek nie jest brany dalej do analizy przecięć z innymi odcinkami.
\end{itemize}
Taki sposób pozwala na ograniczenie zbioru sprawdzanych odcinków do sąsiedztwa potencjalnie przecinających się odcinków.
Dzięki temu nie potrzeba sprawdzać każdej pary odcinków ze sobą, co skutkuje złożonością obliczeniową rzędu O((n+k)log n), gdzie n to liczba odcinków, a k liczba przecięć.

Rozwijając ten algorytm, punkty przecięć zostaną skrzyżowaniami z punktu widzenia działania programu, a drogi na przecięciach zostaną zmodyfikowane w zależności od tego w jakiej proporcji punkt podzielił odcinki.


\subsection{Określenie czy pacjent znajduje się w obszarze Państwa}

//Even-odd rule Algorithm

\subsection{Przeszukiwanie grafu w celu znalezienia miejsca w szpitalu}

//Zmodyfikowany DFS/BFS

\section{Testy oprogramowania}


\section{Źródła}

\end{document}{article}


